Top-level forms:
  (table <id> <policy>...)
  (stream <id> <policy>...)
  (view <id> <policy> ...)
 
Misc parts
  set  : a type of tuple expression
         syntax for building rules,  (set <rule> ...)
         handles filters and tuple construction and let and :* and :- more fluidly than #(...) syntax would
  #(e ...) : tuple expression, returns a tuple, basically (tuple (id ex) ...) but with nicer syntax
             Can be used anywhere an expression can be used
             Names may be omitted, then values are only referred to by index (handy for making lists!)
  rule : (<id> <bind op> <expr>)
  expr : Classic, nested expressions
         -[<id>]>  Post-fix composition (+ named intermediate result)

Policies :
  (meta ...) 
     ;; Builds a set of key/value pairs that are associated with the object
     ;; May not be available at runtime...so try not to depend on it for runtime information
     ;; ...but it becomes important, we'll make it available at runtime...maybe policy-based in some way
  (data <generator>...)
     ;;Build a tuple for 'use' in the current context
     ;;Could be store-in-the-table or put-on-the-stream (for example)
  (fields (<id> ...))
  (fields (<id> : <type> ...))
  (fields (<id> : <expr> ...))   
      ;;Default value provided by expression, maybe need a way to override type inferencer here...
  (retain  <condition>...)
     ;;Only valid on table 
     ;;Any whole-table condition will be executed after per-item conditions
  (render <id> <tuple-expr>)
     ;;Only valid in table definitions, indicates how to render the data
  (GOG <GOG expression>)
    ;;Define a grammar-of-graphics transform on the data
    ;;The GOG expression is not well defined right now
  (guide (<type> <from-table> <for-field> <sample-type> <tuple-expr>) ...)
     ;;Defines a guide 
     ;;If used on a table with "data", the <from-table> must be the
       containing table
  
  (tables [<id> | <table>] ...)
     ;;Only valid in view
     ;;Indicates which tables to render on the view
     ;;Table definitions can be in-lined, if desired
     ;;Table ID's may be paired with a renderer,
        otherwise a default or one specified in the table is used
  (component <bool>)
     ;;Only valid in view
     ;;#f -- This view needs to be its own panel/window/etc
     ;;#t -- This is only used as a component in other views
       


Generators
(when <predicate> <f> <tuple-expr>)
   ;; f is a function that generates tuples
   ;; When the predicate true, execute f and apply the rules to each result of f
   ;; Requires some work for scheduling the predicate...
      probably once for each top-level event
   ;; Other generators are sugar for this (when ...) form
(file <filename> <parser>? (<key> : <value>) ...)
   ;;Filename MUST be supplied
   ;;Parser is something like csv, tab, a regular expression, etc
   ;;Key/value pairs determine many things, for example "Skip", "Sep", "Strict", "Pre"
   ;;Skip is number of lines to skip
   ;;Preprocessor is tuple->tuple where output tuple must match the schema
     of the context or a be sentinel "no tuple" or "error".
     Strict and skip can be implemented via this mechanism (probably).
   ;;(when (start-of-program) (parser filename args) preprocessor)
(push <schema> <tuple-expr>)
   ;;Schema is the tuple expected to be received, but you don't know from where
   ;;(when (not-empty recvQueue) (pop recvQueue) <tuple-expr>)
   ;;Implies that somewhere in the program (push recvQueue) is called
(pull <from> <tuple-expr>)
   ;; Indicate a place to get tuples from.  
   ;; The type of the from determines exactly what that means
   ;; Stream will just get one tuple at a time
   ;; Table will iterate the whole table whenever the table changes in some way
   ;; (when (changed <from>)  (items <from>) <tuple-expr>) 
(init (<tuple*-expr>) <tuple-expr>)
   ;; Generates the given set of values
   ;; (when (start-of-program) (tuples <tuple> ...) <tuple-expr>)


Other notes:
  :*, :- as binding operators
  _ means "default value"
  Guide and render-GOG should be library transforms that compile out to other core forms
